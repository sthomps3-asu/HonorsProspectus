\section{Topic Overview:}

\indent Low-power wearable systems that are capable of continuous image capture can provide portable analysis of a subject's motion through
space and time. These devices would be capable of gathering and processing a large amount of data while the wearer is moving throughout their
day. This motion analysis can be used for dynamical modeling and provide a valuable data set to researchers and physicians that are working
to with patients undergoing physical rehabilitation for mobility disorders. The data collected could offer insights into patient's movement
patterns that may otherwise not be observable in a clinical setting. During physical rehabilitation, it is often difficult for patients to
reinforce their clinical rehabilitation practices outside of a scheduled appointment. A system capable of sensing and interpreting patient's
movement patterns offers the possibility of providing real-time feedback allowing patients to adjust their motion. Such a capable and portable
device can usher in new methods of rehabilitation that are currently not possible.

\indent Image capture sequences are inherently inflexible and power-hungry due to the static frame rate that is set prior to a capture
sequence. Continuously sensing with a constant framerate to obtain an appreciable level of resolution is taxing on the energy systems of the
image sensing device. Such demanding energy requirements make continuous image sensing impractical for compact wearable devices. Designing embedded
control mechanisms that will enable computer vision systems to operate with variable frame rates during continuous image sensing will provide energy
efficiency as well as dynamically variable spatiotemporal resolution. This new approach will enable a new class of wearable devices that are capable
of continuous image sensing. Challenges in energy efficiency and sensor timing will need to be met and overcome to enable a dynamically variable frame
rate that would provide adequate resolution while operating on a tight energy budget. A redesign of the required software stack is necessary to allow
for continuous mobile vision. To fully realize this new class of wearable device, the flexibility to adapt to different spatiotemporal requirements as
the task may need them must be the driving design trajectory throughout this research effort. Allowing the system to continually tradeoff temporal
resolution of image capture for energy efficiency is vital to enable dynamical modeling and meaningful processing of the gathered data.

\indent The applications of highly-portable systems that are capable of continuous image sensing extend beyond medical and rehabilitation applications.


